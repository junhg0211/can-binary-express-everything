\documentclass{article}

\usepackage{kotex}
% \usepackage[margin=1in]{geometry}

\title{2진수의 기계적인 활용: 요약}
\author{2021 동탄고등학교 30925 전한결}

\begin{document}

\maketitle

\paragraph{이 주제를 선정한 이유}
컴퓨터가 0과 1로 이루어져있다는 말은 많이 들은 적이 있지만,
어떻게 0과 1만으로 우리가 편리하게 사용하는 기능을 구현할 수 있었는지 의아했다.
이에 직접 컴퓨터 속에서 0과 1이 어떤 식으로 작동하는지를 알아보고,
컴퓨터의 원리를 2진수적인 측면에서 알아보기로 했다.

\paragraph{2진수와 컴퓨터속 사용}
2진수는 0과 1로만 이루어진 수이다.
우리가 평소에 사용하는 수 체계는 0부터 9까지 총 10개의 숫자로 이루어져 있으므로 10진수이다.
2진수를 통해서는 10진수로 표현할 수 있는 대부분의 것을 표현할 수 있다.

수학자들은 \textit{2진수}로 자료를 표현하는 것이 아니라 \textit{수}로서 자료를 표현하는
방법만을 알아내면 되었던 것이다.
전 세계의 여러 수학자들이 컴퓨터 속에서 사용하는 자료 구조의 여러가지 표현 방법에 대한
표준을 제시했고, 그 중 일부가 운영체제나 프로그래밍 표준 개발자에 의해 선택되어
사용되고 있다.
이 발표에서는 그중 정수, 유리수, 문자열, 영상 자료의 표현방법에 대해 다뤘다.

또한 이러한 자료를 가공하고 실생활에 적용하기 위해서는 실생활 속의 여러 변수를 프로그랭에
적용하여 그에 따른 함숫값을 알아내는 계산 과정이 필요한데, 이 계산은 논리합과 논리 부정으로 이루어진
소프트웨어에서 담당한다.
\textit{컴퓨터 응용 소프트웨어 개발자}란 이러한 소프트웨어를 개발하는 사람을 말한다.

\paragraph{내 생각}
컴퓨터를 공부하고 나서 컴퓨터 속 디지털 자료 처리와 실생활이나 이론적으로 이루어지는 자료의 처리를
비교할 수 있게 되었는데, 이를 통해서 내가 세상을 인식하는 방식에 영향을 받게 되었고, 이는
내가 앞으로 살아가는 데에 있어서 사상의 기반이 되는 철학을 바꾸었다.

기술 발전에 따라 많은 갈등이 발생한다. 뇌파 분석 등의 기술이 발전하면 생길 수 있는
기술 오용 | 뇌파의 조종을 통한 인간의 행동 제어, 윤리적 문제 발생 등이 그것이다.
이러한 문제에서 벗어나기 위해 우리 모두는 기술에 대한 올바른 윤리 의식을 함양할 필요가 있겠으며,
그것을 위해서 전세계적인 기술 사용의 표준을 정하거나 올바른 가치관을 가진 교육자가
많아지도록 하는 등의 노력이 필요할 것이다.

\end{document}
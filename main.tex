\documentclass{article}

\usepackage{kotex}
\usepackage{amsmath}

\title{2진수와 그 기계적인 활용}
\author{2021 동탄고등학교 30925 전한결}

\begin{document}
    
\maketitle

\section{10진수와 2진수}

우리는 살면서 10진수를 통해 숫자를 표현한다.
인류가 굳이 10진수를 사용하는 이유는 정확히 밝혀지지 않았지만,
대부분의 인류학자들은 그것이 사람의 손가락으로 수를 세는 것에서
수가 기원했기 때문이라고 말한다.

10진수는 0부터 9까지의 10가지 모양으로 `없음'에 해당하는 수부터 `아홉'에 해당하는
수까지를 표현한다. 이 숫자를 `일의 자리 숫자'라고 한다.
만약 `아홉' 이상의 수를 표현해야 할 때에는 숫자 옆에 `하나'에
해당하는 수를 쓰고 숫자가 할 수 있는 경우의 수가 한 번 다 한 후에 몇이 더 큰지를
`일의 자리 숫자`에 나타낸다. 이 숫자를 `십의 자리 숫자'라고 한다.
십의 자리 숫자도 일의 자리 숫자와 같이 10가지의 경우의 수를 가질 수 있으며,
이 경우의 수를 모두 소진했을 때에는 또 옆에 숫자를 쓰고, 이 숫자를
`백의 자리 숫자'라고 한다.
우리가 사용하는 10진수에서는 이러한 진행을 이어가며 그 값을 나타낸다.

$$
24
$$

컴퓨터는 기계적인 신호로 이루어져 있고, 기계적인 신호는 전압이 강한지 약한지를 통해
그 정보를 전달한다. 대부분의 기계 장치에서는 전력이 흐르지 않는, 즉 전압이 0인 지점
점과 전압이 최대\footnote{보통은 이 최대점이 어디인지 기계를 설계하는 사람이 정한다.}
인 지점의 두 가지를 $1/2$ 또는 $2/3$지점에서 나누어 구분한다.
즉, 기계장치에서는 전압의 세기로 총 두 가지 경우를 나타낼 수 있는 것이다.

하지만 두 가지 경우로 의미있는 정보를 전달하기는 쉽지 않다. 수학자들은 당연히 이러한
문제를 알고 있었으며, 이 문제를 타파하기 위해 10진수에서 값을 표현하는 방식을 빌려왔다.
우리는 이것을 `2진수'라고 부른다.

10진수의 경우, 사용하는 10개의 글자 중 하나를 써서 표현할 수 있는 상황의 개수가
10개이기 때문에 10가지를 넘는 정보를 표현하기 위해서 숫자 옆에 다른 숫자를 써서
10가지를 초과하는 정보를 표현한다. 2진수도 이 방식을 적용해서 사용할 수 있다.
단지, 수를 구성하는 숫자가 2개밖에 없을 뿐이다.

2진수는 0부터 1에 해당하는 2가지 모양으로 `없음'에 해당하는 수부터 `하나'에
해당하는 수까지를 표현한다. 이 숫자를 10진수에서와 같이 `일의 자리 숫자'라고 한다.
만약 `하나' 이상의 수를 표현해야 할 때에는 숫자 옆에 `하나'에 해당하는 수를 써서
숫자가 할 수 있는 경우의 수가 한 번 다 한 후에 몇이 더 큰지를 `이의 자리 숫자'에
나타낸다. 이 숫자를 `이의 자리 숫자'라고 한다.\footnote{왜냐하면 2진수 $10$에
해당하는 수를 우리가 `이'라고 부르기 때문이다. 10진수의 경우도 $10$에 해당하는
수를 우리가 `십'이라고 부르기 때문에 십의 자리 숫자라고 하는 것이다.}
이와 같은 방식으로 `사의 자리 숫자', `팔의 자리 숫자', `십육의 자리 숫자' 등을
정의하여 수를 표현한다.

$$
11000
$$

위에 $11000$이라고 2진수로 표현한 수는 우리가 $24$라고 하면 아는 수와 같은 값을
나타낸다. 2진수를 보고 그 크기는 어떻게 가늠할 수 있을까?

우리가 10진수를 읽을 수 있는 것은 10진수에 대해 배웠기 때문이다. 2진수도 배운다면
누구나 읽어낼 수 있다. 따라서 일단 10진수에 대해서 다시 한 번 알아보자.

예를 들어, 우리 눈 앞에 세 자리 숫자가 있다고 하자. 이 숫자는 10진수로 표현되었다.

$$
794
$$

이 숫자는 백의 자리 숫자가 7, 십의 자리 숫자가 9, 일의 자리 숫자가 4이다.
이 값은 다음과 같이 표현될 수도 있다.

$$
7 \times 100 + 9 \times 10 + 4 \times 1
$$

또한 수학적으로는 다음과 같은 방식으로 표현할 수도 있다.

$$
7 \times 10^2 + 9 \times 10^1 + 4 \times 10^0
$$

10진수는 가장 오른쪽에 있는 수(일의 자리 숫자)에다가 $10^0$을 곱하고, 그 값에
그 다음 자리에 있는 수(십의 자리 숫자)에다가 $10^1$을 곱한 값을 더하고,
이러한 과정을 숫자가 끝날 때까지 반복한 것이다. 이를 일반화하면 다음과 같다.
n진수는 가장 오른쪽에 있는 수($n^0$의 자리 숫자)에다가 $n^0$을 곱하고, 그 값에
그 다음 자리에 있는 수($n^1$의 자리 숫자)에다가 $n^1$을 곱한 값을 더하고,
그 다음 자리에 있는 수($n^2$의 자리 숫자)에다가 $n^2$을 곱한 값을 더하고\dots
결국 $n$에 대한 다항식으로 표현할 수 있는 수를 말한다.
그렇다면 2진수 $11000$은 다음과 같은 (2에 대한) 다항식으로 표현할 수 있다.

$$
\begin{aligned}
     & 1 \times 2^4 + 1 \times 2^3 + 0 \times 2^2 + 0 \times 2^1 + 0 \times 2^0 \\
    =& 1 \times 16 + 1 \times 8 + 0 \times 4 + 0 \times 2 + 0 \times 1 \\
    =& 1 \times 16 + 1 \times 8 \\
    =& 16 + 8 \\
    =& 24
\end{aligned}
$$

다시 수학자들의 이야기로 돌아와보자. 수학자들은 이렇게, $n$진수로 표현한 숫자가 결국
$n$에 대한 다항식으로 표현 가능한 수라는 것을 발견해내었다. 2진수 숫자를 하나의
전기 신호로 표현하고, 그 전기 신호를 여러개 작성하면 더 다양한 값을 나태낼 수 있다는
사실을 알아낸 것이다. 전선을 여러개 사용하는 방법과 시간차를 두고 여러 값을 보내는
방식 중에서 수학자들은 시간차를 두고 여러 값을 보내는 방식을 선택했다. 컴퓨터와 같은
기계장치들은 시간차를 두고 전송되는 일련의 신호를 일련의 방식에 따라 이해하고 처리한다.

\section{2진수 연산}

컴퓨터가 2진수를 사용하여 숫자를 전송하고 처리한다는 사실을 알게 되었다.
그렇다면 어떤 

\end{document}